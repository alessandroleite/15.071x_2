
Our last example involves a company called D2Hawkeye,
a medical software company founded in 2001.
The company combined data with analytics
to improve quality and cost management in health care.
In 2009, the company was analyzing 20 million people
monthly.
It is impossible for humans to sift through patient records
and assess quality and cost without algorithms.

This motivated the approach that D2Hawkeye took.
Let us discuss the data that D2Hawkeye utilized.
Health care industry is data rich,
but data may be hard to assess.
It is often unstructured and unavailable.
The company used insurance data regarding procedures,
prescriptions, and diagnosis.
It further defined new risk factors
based on doctor's insights.
For example, obesity and depression.
Finally, it used demographic information,
like gender and age.

What were the analytics used?
The goal was to predict future heath care costs,
and identify high-risk patients to be
prioritized for intervention.
The company created interpretable models
for doctors to analyze and verify.
This led to significant improvements
over just using historical costs.

So what is the edge?

The analytics used led to substantial improvement
in D2Hawkeye's ability to identify
patients who need more attention.
The approach allowed expert knowledge
to identify new variables and refine existing variables.
Further, it allowed the ability to make predictions
for millions of patients without manually
reading patient's files.