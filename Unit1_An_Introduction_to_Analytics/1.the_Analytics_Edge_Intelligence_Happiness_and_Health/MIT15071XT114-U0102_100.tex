
Our first example today is a story of IBM Watson.
IBM research, a distinguished industrial laboratory,
strives to push the limits of science.
In the mid-1990s it created Deep Blue, a computer that
beat Garry Kasparov, the world champion in Chess at the time,
showing to the world that machines can beat humans
in tasks that people thought were
restricted to human intelligence.
In the 1990s-- in the late 1990s--
it created Blue Gene, a computer to map the human genome.
In 2005 IBM decided to create a computer that
would compete at Jeopardy, a popular game show.

From the T.J. Watson Research Center
in Yorktown Heights, New York, this
is Jeopardy-- The IBM Challenge.
Please welcome back our contestants.
He has never been defeated.
And his winnings of more than \$3.2 million
make him Jeopardy's all time biggest money winner.
From Los Angeles, California, here's Brad Rutter.
An IBM computer system able to rapidly understand and analyze
natural language, including puns, riddles,
and complex questions across a broad range of knowledge,
please welcome Watson.
In 2004 he captivated America by winning 74 consecutive matches
and \$2.5 million on Jeopardy.
From Seattle, Washington, here's Ken Jennings.
And now here is the host of Jeopardy, Alex Trebek.
Thank you, Johnny.
Before we get into the game, there
are just a couple of things I need
to tell you about this match.
Now as I said earlier, Watson will receive the clues
electronically as a text file at the same moment the clues are
revealed to Ken and Brad.
And at the same time I read them.
This competition will be a two game
total point exhibition match.
However, these two games will be played out over the next three
days so we can tell the full story.
Throughout the games you'll get a glimpse of the thinking
process, if you will, that is behind Watson's responses.
Now this will be done through an answer panel display
at the bottom of the screen.
Let's play Jeopardy.
Here we go.

Our first round of play contains these categories--
Literary Character APB, All Points Bulletin; Beatles
People; Olympic Oddities; Name the Decade;
Final Frontiers; and Alternate Meanings.
A little while ago we had a drawing
to determine which player would select first.
Brad, you won that.
So if you're ready make your first choice.
Let's take alternate meanings for \$200, Alex.
Four letter word for a vantage point or a belief.
Brad?
What is a view?
Yes.
Alternate meanings, \$400.
4-letter word for the iron fitting
on the hoof of a horse or a card-dealing box in a Casino.
Watson?
What is shoe?
You are right.
You get to pick.
Literary Character APB for \$800.
Answer the Daily Double.
Now Watson, although you have but \$400,
you know of course that you can risk up
to the maximum value of a clue on the board.
And that is \$1,000.
\$1,000, please.
All right.
Here is the Daily Double clue for you.
Wanted for killing Sir Danvers Carew.
Appearance pale and dwarfish.
Seems to have a split personality.
Who is Hyde?
Hyde, yes.
Dr. Jekyll and Mr. Hyde, either one acceptable.
You're now in the lead with \$1400.
Go again.
Beatles People for \$200.
And any time you feel the pain, hey, this guy, refrain.
Don't carry the world upon your shoulders.
Watson?
Who is Jude?
Yes.
Olympic Oddities for \$200.
Milorad Cavic almost upset this man's perfect 2008 Olympics,
losing to him by one hundredth of a second.
Watson?
Who is Michael Phelps?
Yes.
Go.
Name the Decade for \$200.
Disneyland opens and the peace symbol is created.
Ken.
What are the '50s?
Yes.
Final Frontiers for \$1,000, Alex.
Tickets aren't needed for this event,
a black hole's boundary from which matter cannot escape.
Watson?
What is event horizon?
Yes.
Why is Jeopardy hard?
Jeopardy asks the contestants to answer cryptic questions
in a huge variety of categories.
It is generally seen as a test of human intelligence,
reasoning and cleverness.
No links to the outside world are permitted.
And new questions and categories are created for every show.

Watson is a supercomputer with 3,000 processors
and a database of 200 million pages of information.
It has a massive number of data sources like encyclopedias,
texts, manuals, magazines, Wikipedia, etc.
IBM researchers who developed Watson
used over 100 different analytical techniques
for analyzing natural language, finding candidate answers
and selecting the final answer.
We'll discuss this more later in the class.
In February, 2011, a two game exhibition match
aired on television six years after the initial conception
of the idea to build Watson.
Watson competed against the best two human players of all time
and challenged the meaning of intelligence.
Now Watson is being used for many applications including
selecting the best course of treatment for cancer.
What is the edge in Watson?
Watson combined many algorithms to increase
accuracy and confidence.
We'll cover many of them in this class.
IBM approached the problem in a different way
than how a humans does it.
Watson deals with massive amounts of data,
often in unstructured form, which is important
as 90\% of the data in the world is unstructured.