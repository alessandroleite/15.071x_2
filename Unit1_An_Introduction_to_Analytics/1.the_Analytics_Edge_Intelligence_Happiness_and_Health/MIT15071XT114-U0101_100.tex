
Welcome to 15.071x, The Analytics Edge.
In this first lecture, we'll discuss how analytics redefines
the meaning of intelligence, how it can contribute
to your personal happiness and health.
Data is transforming business, social interactions,
and the future of our society.
The amount of electronic data that exists in the world today
is a phenomenal 2.7 zettabytes, which
is equal to the storage required for more than 200
billion high-definition movies.
Not only the amount of data is extremely high,
but it increases at exponential rates.
Our ability to effectively process this data
is also increasing very rapidly.
As an example, decoding the human genome
originally took 10 years to process.
Now it can be achieved in just one week.

Analytics is increasingly important in the world today,
and this influence is expected to increase.
McKinsey estimates that there is a shortage
of 140,000 to 190,000 people with deep analytical skills
to fill the demand of jobs in the United States by 2018.
IBM has changed its business focus over the last 100 years
very successfully from typewriters to mainframes
to personal computers to consulting, and now
to analytics.
It has invested over \$20 billion since 2005
to grow its analytics business.
Companies will invest more than \$120 billion
by 2015 on analytics, hardware, software, and services.
Analytics is becoming increasingly critical
in almost every industry, from health care,
to media, to sports, to finance, to government, and many others.
Let us give a definition of analytics
so we make it as concrete as possible.
We define analytics to be the science of using data
to build models that lead to better decisions, that
add value to individuals, to companies, to institutions.
Note that there are four ingredients-- data, models,
decisions, and value.
And all four are needed in this definition.

What are the key messages of this class?
First, analytics provides a competitive edge
to individuals and companies.
Analytics are often critical to the success of a company.
And they provide often the decisive essential technology.
Our teaching methodology is to teach you analytics techniques
through real-world examples and real data.
And our overarching goal is to convince you of the analytics
edge and inspire you to use analytics
in your career and your life.
The teaching team comes from the Operations Research
Center at MIT and the Sloan School of Management.
I am Dimitris Bertsimas.
I have received my Ph.D. from MIT from 1985 to 1988.
And I have been with the MIT faculty
at the Sloan School of Management since 1988.
Currently, I'm the co-director of the Operations Research
Center.
My career is centered in analytics.
And I believe that analytics can change the world.
The other instructor of this class is Allison O'Hair.
Allison received her Ph.D. from the Operations Research Center
at MIT in 2013.
Allison and I have worked together
in the area of health care analytics,
and are working at the moment with our colleague Bill
Pulleyblank on an analytics textbook.
The teaching assistants in the class
are Iain Dunning, Angie King, Velibor Misic, John Silberholz,
and Nataly Youssef, all Ph.D. students at the Operations
Research Center at MIT.

To give you a sense of the breadth of applications
in this class, we'll cover the story
of IBM Watson, the computer that beat the best human players
in Jeopardy!, the company eHarmony, the Framingham Heart
Study, and D2Hawkeye, a company that I have
been involved for almost a decade.
Other examples include the story of Moneyball,
how analytics can help predict the Supreme Court decisions,
the role analytics have played in predicting
the outcomes of the US presidential elections, how
analytics can utilize effectively data from Twitter,
Netflix, airline revenue management, radiation
therapy, sports scheduling, and many others.