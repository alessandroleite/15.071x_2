
Our third example is about the Framingham Heart Study.
This study represents one of the most important studies
of modern medicine.
It is an ongoing study of the residents
in Framingham, Massachusetts.
It started in 1948, and is now on the third generation.
Much of the now-common knowledge regarding heart disease
came from this study.
For example, the fact that high blood pressure
should be treated.
Clogged arteries are not normal.
Cigarette smoking can lead to heart disease.
Let us give some statistics about heart disease.
Heart disease has been the leading cause of death
worldwide since the 1920s.
7.3 million people died from coronary heart disease in 2008.
Since 1950, age-adjusted death rates have declined 60%.
In part, due to the results of the Framingham Heart Study.
What is the data in this study?
There were 5,209 patients enrolled in 1948.
The patients were given a questionnaire and exams
every two years, measuring their physical characteristics,
their behavioral characteristics,
and medical test results.
The patient population, the exams, and the questions
expanded over time.
The approach the Framingham Heart Study utilized
was a regression to predict whether or not
a patient would develop heart disease in the next 10 years.
The model tested and adjusted for different populations.
The results of the study are available online
so users can calculate their risk of heart disease
based on total cholesterol, HDL, and systolic blood pressure.

So what is the edge?
It provided necessary evidence for the development of drugs
to lower blood pressure.
The study further paved the way for other cliniical prediction
rules that predict clinical outcomes using patient's data.
Finally, the study demonstrated how
a model allows a medical professional
to make predictions for patients worldwide.